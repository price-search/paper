\section{Ferramentas}
\label{sec:ferramentas}

\subsection{Angular}
\label{ssec:Angular}
Angular\cite{afonso2018angular} é uma plataforma e \textit{framework} para construção de interface de aplicações usando HTML (do inglês, \textit{HyperText Markup Language}), CSS (do inglês, \textit{Cascading Style Sheets}) e, principalmente, JavaScript, criada pelos desenvolvedores do Google. 

Ele possui basicamente dois tipos: AngularJs e o Angular 2+; que são tecnologias completamente diferentes. O AngularJs é um framework baseado totalmente em JavaScript para desenvolvimento web e o Angular 2+ que surgiu para desenvolvimento não só web, mas também para mobile, usa o TypeScript\footnote{Extende o Javascript adicionando tipos para a linguagem. Disponível em \url{https://www.typescriptlang.org/}.}, e que teve várias mudanças no modo de construção da aplicação. A arquitetura do Angular permite organizar a aplicação por módulos através dos \textit{NgModules}\footnote{Configuram o injetor e o compilador e ajudam a organizar as coisas relacionadas juntas. Disponível em: \url{https://angular.io/guide/ngmodules}.}, que fornecem um contexto para os componentes serem compilados. 

Uma aplicação sempre tem ao menos um módulo raiz que habilita a inicialização e, normalmente, possui outros módulos de bibliotecas. Os componentes deliberam as visualizações — que são conglomerados de elementos e funcionalidades de tela — que o Angular modifica de acordo com a lógica e os dados da aplicação[2]. O Angular também possui uma ferramenta chamada Angular CLI (do inglês, \textit{Command Line Interface}), que permite criação de utensílios que auxiliam na criação do projeto, como: componentes, serviços, interfaces, etc. Segue abaixo algumas vantagens do uso do Angular:  

\begin{itemize}
    \item Produtividade: desenvolver uma aplicação com ele requer bem menos código, sendo inclusive intuitivo e dando suporte para a modularização; 
    \item Fácil manuseamento: muito fácil entender o funcionamento das aplicações lendo apenas o HTML;   
    \item Criação de \textit{frameworks}: é possível criar nosso próprio \textit{framework} a partir dele;   
    \item Teste unitário: é simples, pois toda a estrutura é desacoplada e as dependências são injetadas, o que facilita a criação de fakes\footnote{ São implementações reais e funcionais de alguma dependência, mas de alguma forma são incompletas para serem colocadas em produção.}, stubs\footnote{É um pedaço de código usado para substituir algumas outras funcionalidades de programação.}, spies\footnote{É uma denominação dada a um objeto que grava suas interações com outros objetos.} e mocks\footnote{Objeto que imita um objeto real para teste}, melhorando -- e muito -- todo o processo de teste de controladores, serviços e diretivas. 
\end{itemize}

\subsection{TypeScript}
\label{ssec:TypeScript}
Para a parte de \textit{back-end} escolhemos o TypeScript\footnote{Disponível em: \url{https://www.typescriptlang.org/}.} desenvolvida pela Microsoft, que é uma linguagem de programação código aberto e traduzida, ou seja, no momento de pré-execução da aplicação, o TypeScript é traduzido para JavaScript, que é uma linguagem de programação interpretada. Tendo como uma das maiores diferenças a tipagem de variáveis, é possível um código em TypeScript ser facilmente convertido em JavaScript. 

Seu uso elimina muitos erros gerados aleatoriamente que resultam de tipos de dados não confiáveis. Graças a isso, o programador pode se concentrar no desenvolvimento de um aplicativo e não perde tempo desnecessário na pesquisa e análise dos erros resultantes. Outra vantagem do TypeScript é a capacidade de eliminar a falta de consistência entre a resposta real do servidor e o formato de dados esperado pelo \textit{front-end}.\cite{Jakub2019TypeScript}

\subsection{NestJS}
\label{ssec:NestJS}
O NestJS\footnote{Disponível em: \url{https://nestjs.com/}.} é um \textit{framework} para criação de \textit{back-end} web usando TypeScript, para desenvolvimento de REST APIs (do inglês, \textit{Application Programming Interface}), com estruturas de código similar ao Angular, ele cria uma camada de abstração em cima de uma das bibliotecas mais utilizadas para criação de aplicações web no mundo, o Node.js.  Então é muito difícil comentar sobre um, sem mencionar o outro, é preciso ressaltar que deve-se ter instalado em sua máquina o Node.js para que seja possível utilizar do NestJS. Ele é uma estrutura para criar aplicativos de forma eficiente, ele conta com um CLI que permite a criação e configuração inicial do projeto.

O NestJS fornece uma arquitetura de aplicativos pronta para uso que permite que desenvolvedores e equipes criem aplicativos altamente testáveis, escaláveis, com acoplamentos frouxos e de fácil manutenção \cite{kamil2020nestjs}. Seguem abaixo algumas vantagens do NestJS .

\begin{itemize}
    \item A estrutura de pastas no NestJS é fortemente baseada em Angular; 
    \item A estrutura é muito orientada a anotações, e as anotações tornam o desenvolvimento mais simples;   
    \item Ferramenta de linha de comando permitirá monitorar o projeto, gerar componentes da arquitetura NestJS e exibir informações do projeto.
\end{itemize}

\subsection{Visual Studio Code}
\label{ssec:VSCode}
Como IDE (\textit{Integrated Development Environment}) a ferramenta escolhida para ser utilizada foi o Visual Studio Code\footnote{Disponível em: \url{https://code.visualstudio.com/}.} que é um editor de código-fonte desenvolvido pela Microsoft para Windows, Linux e macOS. A escolha dessa ferramenta foi devido ela ser código aberto, possuir suporte para o Git e a abrangência que ela traz com o uso de extensões, ou seja, ferramentas que estendem e facilitam na programação para a devida linguagem escolhida.

Com suporte para centenas de idiomas, o Visual Studio Code ajuda o usuário a ser instantaneamente produtivo com realce de sintaxe, correspondência de colchetes, recuo automático, seleção de caixa, trechos e muito mais. Atalhos de teclado intuitivos, personalização fácil e mapeamentos de atalhos de teclado contribuídos pela comunidade permitem navegar com facilidade pelo seu código. \cite{mjbvz2020VSCode}


\subsection{Gitpod}
\label{ssec:Gitpod}
Como o GitHub\footnote{Disponível em: \url{https://github.com/}.} possui recursos limitados para edição, forçando muitas vezes os usuários utilizar de recursos locais em suas máquinas para resolver até pequenos problemas, o Gitpod\footnote{Disponível em: \url{https://www.gitpod.io/}.} surge como uma luz para os amantes de GitHub. Ele não é simplesmente uma IDE online, ele é muito mais do que isso, ele é como uma extensão para edição do GitHub, usando como IDE base o Visual Studio Code. 

Diferentemente das IDEs tradicionais da nuvem e da área de trabalho, o Gitpod entende o contexto e prepara o IDE automaticamente. Por exemplo, se você estiver criando um espaço de trabalho Gitpod a partir de um \textit{pull request}\footnote{É um mecanismo que é usado para integrar uma mudança de código de um projeto, possibilitando a revisão de código} do GitHub, a IDE será aberta no modo de revisão de código. Além disso, os espaços de trabalho do Gitpod devem ser descartáveis. Ou seja, você não precisa manter nada. Eles são criados quando você precisar deles, e você pode esquecê-los quando terminar \cite{jankeromnes2020Gitpod}. 


\subsection{Swagger}
\label{ssec:Swagger}
A ferramenta Swagger\footnote{Disponível em: \url{https://swagger.io/}.} é feita para modelagem, documentação e geração de código para REST APIs, seja manualmente ou geradas automaticamente a partir do código-fonte, traz uma enorme vantagem para quem deseja trabalhar com testes de APIs. A capacidade das APIs de descrever sua própria estrutura é a raiz de toda a grandiosidade do Swagger.

Podemos criar automaticamente uma documentação API bonita e interativa. Também podemos gerar automaticamente bibliotecas de clientes para sua API em vários idiomas e explorar outras possibilidades, como testes automatizados. O Swagger faz isso solicitando que sua API retorne um YAML\footnote{É um formato de arquivo. Disponível em: \url{https://yaml.org/}.} ou JSON\footnote{JSON, um acrônimo de JavaScript Object Notation, é um formato compacto, de troca de dados simples e rápida entre sistemas. Disponível em: \url{https://www.json.org/json-pt.html}} que contenha uma descrição detalhada de toda a API. Este arquivo é essencialmente uma lista de recursos da sua API que adere à especificação \textit{OpenAPI} \cite{shockey2020Swagger}. A especificação solicita que você inclua informações como:

\begin{itemize}
    \item Quais são todas as operações suportadas pela sua API? 
    \item Quais são os parâmetros da sua API e o que ela retorna?   
    \item Sua API precisa de alguma autorização?
    \item Termos, informações de contato e licença para usar a API.
\end{itemize}

\subsection{Ambiente}
\label{ssec:Ambiente}
Com as principais ferramentas já descritas, existe outras que também contribuem para o desenvolvimento, e ajudam nas configurações de ambiente que serão descritas a seguir. Entre elas estão o Node.js, o interpretador JavaScript, usado para executar os códigos transpilados a partir do TypeScript. É preciso tê-lo instalado para que se consiga rodar os comandos NestJS.

Para controle de versão utilizamos o Git com o código hospedado no GitHub, utilizados para revisão de código (através dos \textit{Pull Requests}), gerenciamento de tarefas (com o \textit{Issues}) e integração contínua (através do GitHub Actions\footnote{Permite criar fluxos de trabalho personalizado de ciclo de vida de desenvolvimento do software. Disponível em: \url{https://github.com/features/actions}}). 

E por fim o banco de dados escolhido foi o PostgreSQL \footnote{Sistema gerenciador de banco de dados objeto relacional, desenvolvido como projeto de código aberto. Disponível em \url{https://www.postgresql.org/}.}, por se tratar de um \textit{software} gratuito e de código aberto, altamente aprovado em ambientes de produção. O PostgreSQL também é altamente extensível, ou seja, você pode definir seus próprios tipos de dados, criar funções personalizadas e até escrever código de diferentes linguagens de programação sem recompilar seu banco de dados. Foi utilizado o TypeORM\footnote{Disponível em: \url{https://typeorm.io/#/}} para fazer o mapeamento entre o código e o banco de dados, com isso, não precisando se preocupar com a escritas de consultas SQL\footnote{É a linguagem de pesquisa declarativa padrão para banco de dados relacional.} (do inglês, \textit{Structured Query Language}).
