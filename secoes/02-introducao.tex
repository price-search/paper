\section{Introdução}
\label{sec:introducao}

Com a alta demanda de compras \textit{online} diariamente, vê-se a dificuldade de encontrar algum mecanismo de comparação e pesquisa de produtos em estabelecimentos locais de diversos tipos. A ideia da aplicação é facilitar o acesso a esses dados e transmitir para o usuário informações que compreendam a sua necessidade.

Foi desenvolvido um aplicativo PWA (do inglês, \textit{Progressive Web App}), que é um termo usado para aplicativos \textit{web} que funcionam como aplicativos móveis, que combinam tecnologias para oferecer melhores recursos aos usuários, que não precisarão baixar o aplicativo em seu dispositivo para acessá-lo. Esse tipo de aplicativo visa combinar vários recursos oferecidos pelos navegadores com benefícios de experiência móvel. \cite{souza2017pwa}

A crise mundial de saúde ocasionada pela COVID-19 trouxe diversos desafios para a população, especialmente para os estabelecimentos pequenos e/ou locais e seus consumidores, como por exemplo a necessidade novas maneiras de se evitar o contato físico e aglomeração de pessoas. Atualmente ainda é visível como esse público não detém uma solução adequada para o problema e, por isso, nota-se a importância de evitar agrupamentos de pessoas e acessos simultâneos aos mercados.

A aplicação entra justamente para evitar tais situações, e ainda fornece informações necessárias aos consumidores para encontrar o melhor preço e também aos vendedores, aumentando sua percepção de venda dos produtos. Um estudo recente sobre as reações de alguns compradores em relação às ofertas de preço mostrou que ``às vezes, se eu conseguir um bom negócio no suporte de descontos, me sinto bem com isso, e  ao passear pelo restante da loja ou do shopping, não me sinto culpado se comprar itens mais caros''. \cite{grewal1998effects}

Para o consumidor, a ferramenta vai trazer suporte quanto às escolhas dos produtos, enfatizando suas informações e comparações entre estabelecimentos onde esse produto se encontra, e aumentando a sua percepção de compra e valor, no qual nota-se que os clientes equilibram os benefícios da compra com os custos \cite{grewal1998effects}. Os vendedores devem entender que existem diversos fatores que afetam a compra de um determinado produto, e onde o produto se encaixa, variando desde satisfazer necessidades exclusivas até demandas comuns. \cite{grewal1998effects}
