\section{Introdução}

Com a alta demanda de compras online diariamente, vê se a dificuldade de encontrar algum mecanismo de comparação e pesquisa de produtos em estabelecimentos locais (pequenos estabelecimentos de diversos tipos). A ideia da aplicação é facilitar o acesso a esses dados, e transmitir para o usuário várias informações que compreendam a sua necessidade.

Foi desenvolvido um aplicativo PWA (do inglês, \textit{Progressive Web App}), que é um termo usado para aplicativos web que funcionam como se fossem aplicativos móveis, que combina tecnologias para oferecer melhores recursos aos usuários, no qual eles não precisem baixar o aplicativo em seu dispositivo para acessá-lo. Esse tipo de aplicativo tenta combinar vários recursos oferecidos pelos navegadores com benefícios de experiência móvel. {\cite{souza2017pwa}}

E como a crise mundial de saúde ocasionada pela COVID-19 trouxe diversos desafios para a população, especialmente para os estabelecimentos pequenos e/ou locais e seus consumidores, novas maneiras de se evitar o contato físico e aglomeração de pessoas se mostraram necessárias, no entanto, ainda é visível como esse público não detém uma solução adequada para o problema, com isso, nota-se a importância de evitar aglomerações e acessos simultâneos aos mercados.

A aplicação entra justamente para evitar tais atos, e ainda fornece informações necessárias aos consumidores para encontrar o melhor preço e também aos vendedores, aumentando sua percepção de venda dos produtos. Um estudo recente sobre as reações de alguns compradores em relação as ofertas de preço mostrou o seguinte: ``Às vezes, se eu conseguir um bom negócio no suporte de descontos, me sinto bem com isso e passear pelas outras partes [do shopping ou da loja] e não me sinto culpado se comprar itens mais caros e com preços originais''.{\cite{grewal1998effects}} 

Para o consumidor a ferramenta vai trazer suporte quanto às escolhas dos produtos, enfatizando suas informações e comparações entre estabelecimentos onde esse produto se encontra, e aumentando a sua percepção de compra e valor, no qual nota-se que os clientes equilibram os benefícios da compra com os custos{\cite{grewal1998effects}}. Os vendedores devem entender que existem diversos fatores que afetam à compra de um determinado produto, e onde o produto se encaixa, variando de satisfazer necessidades exclusivas a satisfazer necessidades indiferenciadas.{\cite{grewal1998effects}}



