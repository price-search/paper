\section{Conclusão}
\label{sec:conclusao}

Conclui-se que a aplicação Price Search é capaz de solucionar os principais problemas abordados no artigo, estes que seriam a dificuldade de pesquisa de preços em estabelecimentos locais, as quais são a ausência de um lugar centralizado, que implica em pesquisas individuais por cada estabelecimento, como através de ligações ou até mesmo visitas presenciais. 

É possível ver como a aplicação também ajuda bastante na resolução dos problemas temporários trazidos com a crise do COVID-19. Com o Price Search as pessoas deixariam de visitar esses estabelecimentos somente para realizar cotações, ou para verificar a disponibilidade de algum produto, ou até mesmo para agilizar as visitas, já que eles só iriam para pegar os produtos que já foram decididos em casa. Com o Price Search, muitos estabelecimentos que fornecem serviço de entrega poderiam ser mais amplamente utilizados, ao entregar para os utilizadores e para os comerciantes uma maneira muito mais clara e simples de se realizar um orçamento de determinados produtos.

Para o futuro, pensamos em deixar a aplicação mais completa onde você possa realizar tudo por ela. Além de saber onde cada produto é mais barato, o pedido e a compra poderiam ser feitos pelo aplicativo.

A aplicação ainda não foi implantada em produção, não houve nenhum desenvolvimento a respeito de integrações com os diferentes sistemas desses estabelecimentos locais. Serão necessários adaptadores capazes de abstrair as informações de todas os possíveis sistemas, como o banco de dados dos produtos, a disponibilidade deles e seus preços. E no caso de haver uma função de realizar pedido, também será necessário criar uma integração com o atendimento do estabelecimento e com as formas de pagamento.

Uma dificuldade encontrada no desenvolvimento do \textit{back-end} foi a falta de suporte para MongoDB da biblioteca \texttt{@nestjsx/crud}. Por conta disso, houve-se a necessidade de escrever todos os \textit{endpoints} genéricos de operações CRUD manualmente, o que caso contrário poderiam ser gerados de forma automática. Já no \textit{front-end} a dificuldade foi a de aprender uma nova linguagem de programação, e como agregar todo o conhecimento disperso que adquirimos durante o desenvolvimento em uma aplicação.