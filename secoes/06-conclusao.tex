\section{Conclusão}
\label{sec:conclusao}

Após o desenvolvimento da aplicação Price Search, foi possível concluir que o aplicativo funciona de maneira simples e de fácil entendimento e que poderá resolver os assuntos abordados no artigo, que é a dificuldade de pesquisar preços em estabelecimentos locais devido à ausência de um lugar que centraliza este tipo de informação, implicando na necessidade de realizar pesquisas individuais por cada estabelecimento, seja por meio de ligações telefônicas ou até mesmo visitas presenciais.

É possível ver como a aplicação também ajuda na resolução dos problemas temporários trazidos com a crise do COVID-19. Com o Price Search, os usuários não precisam mais visitar estes estabelecimentos para verificar a disponibilidade de algum produto e, em seguida, comparar as diversas ofertas encontradas, já que eles só iriam para retirar os produtos que já foram decididos em casa. Com o Price Search, muitos estabelecimentos que fornecem serviço de entrega poderiam ser mais amplamente utilizados, pois a ferramenta proporciona aos utilizadores e aos comerciantes uma maneira mais clara e simples de se realizar um orçamento de determinados produtos.

Para o futuro, propõe-se estender as funcionalidades da aplicação, permitindo realizar o processo de compra fim-a-fim, ou seja, desde a descoberta do melhor preço de um ou mais produtos até a realização de cotações, pedidos e pagamento.

Não houve esforço no sentido de coleta de dados através da API do Cadastro Nacional dos Produtos ou a respeito do desenvolvimento dos adaptadores responsáveis por realizar a integração com os diferentes sistemas usados pelos estabelecimentos locais. A versão atual da aplicação usa dados falsos criados manualmente a fim de mostrar seu funcionamento. No caso da realização de um pedido, também será necessário criar uma integração com o atendimento do estabelecimento e com as diversas formas de pagamento disponíveis no mercado (cartão de crédito, boleto, transferência bancária, etc.).

Propõe-se também para o futuro, o desenvolvimento dos diferentes tipos de testes automatizados em um ambiente de integração contínua. Atualmente, somente a análise sintática é executado, através do GitHub Actions.

Uma dificuldade encontrada no desenvolvimento do \textit{back-end} foi a falta de suporte para MongoDB\footnote{Banco de dados não relacional orientado a documentos. Disponível em \url{https://www.mongodb.com/}.} pela biblioteca \texttt{@nestjsx/crud}. Tal fato demandaria a necessidde de escrever todos os \textit{endpoints} genéricos de operações CRUD manualmente e, portanto, foi decidido usar o PostgreSQL, suportado pela biblioteca. No \textit{front-end} aaprender uma nova linguagem de programação e, de um modo geral, o desafio de agregar todo o conhecimento adquirido durante o desenvolvimento da aplicação.

Todos os repositórios de código-fonte podem ser encontrados na \href{https://github.com/price-search}{organização do Price Search no GitHub}.
