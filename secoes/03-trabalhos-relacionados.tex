\section{Trabalhos Relacionados}
\label{sec:trabalhos-relacionados}

No mundo cotidiano, vê-se a grande disponibilidade e ofertas de produtos em vários tipos de estabelecimentos, sejam eles: lojas, mercados, farmácias e pequenas lojas que vendem os mais diversos tipos de produtos. Uma ferramenta que pode ser usada para verificar quais são as melhores ofertas de produtos são de sites de busca.

Um dos motores de busca mais usado no Brasil, é o Buscapé\footnote{Portal de serviços gratuitos de busca de produtos e pesquisa de preços. Disponível em \url{https://www.buscape.com.br/}.}, fundado em junho de 1999, empresa que foi fundada com aproximadamente R\$ 4.800,00, hoje possui um faturamento anual de R\$ 300.000.000,00 \cite{EmídiaFelipe2017BUSCAPÉ}. Uma das ideias iniciais para a aplicação, foi quando Rodrigo (um dos criadores do Buscapé), estava procurando uma impressora para comprar na internet, então ele entrou em alguns \textit{sites} de busca da época e encontrou de tudo, menos o preço da impressora; então foi ai que decidiram explorar a ideia de criar um \textit{site} que ajudasse a responder questões típicas de decisões de compra \cite{Arruda2011Buscapé}. O Buscapé passou por várias transformações ao longo do tempo em termos de negócio, e hoje, o site recebe mensalmente 60 milhões de visitas, compara mais de 25 milhões de produtos vendidos por 8.500 lojas, segundo dados da empresa que hoje é líder em comparação de preços no Brasil \cite{HELOÍSA2017Startups}.

Outra ferramenta também conhecida é o Google Shopping\footnote{É um serviço Google para pesquisa e comparação de produtos em lojas \textit{online}. Disponível em \url{https://www.google.com/shopping}} que foi criado pelo Craig Nevill-Manning. Inicialmente a ferramenta realizava pagamentos para os donos de lojas através do Google Ads\footnote{É o principal serviço de publicidade da Google. Disponível em: \url{https://ads.google.com/intl/pt_BR/home/}}, porém em 2012 foi feita uma mudança na ferramenta para que as lojas pagassem ao Google para aparecer no Google Shopping. Com isso, o Google passou a divulgar somente aquelas lojas que pagam para o serviço, e aquelas que não mais o pagavam, não tiveram suas informações divulgadas no site.

Com o passar do tempo, da evolução tecnológica e com novas necessidades dos consumidores, surgiram varias ramificações deste tipo de serviço: o de comparar preços. Existem vários setores diferentes, como por exemplo o site Trivago\footnote{Motor de busca e comparador de preço de hotéis e periféricos em geral. Disponível em: \url{https://www.trivago.com.br/}}, que faz a comparação de preços de hotéis, buscando o valor de diárias de um mesmo hotel em diferentes \textit{sites}, e mostrando a diferença de valor para o usuário poder escolher aquele que oferece o menor preço.

Com a análise feita em diversos aplicativos diferentes, podemos ver que cada um possui alguma característica que difere um do outro, podendo ser o tipo de produto ou também a necessidade. Com isso, a aplicação proposta, denominado Price Search também traz uma diferença dos demais, que é a busca de produtos para estabelecimentos locais, ou seja, se a pessoa pesquisa por exemplo, um saco de arroz, o aplicativo irá retornar todos os estabelecimentos da sua cidade que estão cadastrados no aplicativo com os devidos preços. Sendo assim, os usuários conseguirão saber em qual estabelecimento sua compra saíra com um menor custo sem sair de casa.