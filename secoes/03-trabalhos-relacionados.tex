\section{Trabalhos Relacionados}

No mundo cotidiano, vê-se a grande disponibilidade e ofertas de produtos em vários tipos de estabelecimentos, sejam eles: lojas, mercados, farmácias e pequenas lojas que vendem os mais diversos tipos de produtos. Uma ferramenta que pode ser usada para verificar quais são as melhores ofertas de produtos são de sites de busca.

Um dos motores de busca mais usado no Brasil, é o 'BuscaPé', fundado em junho de 1999, empresa que foi fundada com aproximadamente 4.800 reais, hoje possui um faturamento anual de 300 milhões de reais \cite{EmídiaFelipe2017BUSCAPÉ}. Uma das ideias iniciais para a aplicação, foi quando Rodrigo (um dos criadores do 'BuscaPé'), estava procurando uma impressora para comprar na internet, então ele entrou em alguns sites de busca da época e encontrou de tudo, menos o preço da impressora; então foi ai que decidiram explorar a ideia de criar um site que ajudasse a responder questões típicas de decisões de compra \cite{Arruda2011BuscaPé}. O BuscaPé passou por várias transformações ao longo do tempo em termos de negócio, e hoje, o site recebe mensalmente 60 milhões de visitas, compara mais de 25 milhões de produtos vendidos por 8.500 lojas, segundo dados da empresa que hoje é líder em comparação de preços no Brasil \cite{HELOÍSA2017Startups}.

Outra ferramenta também conhecida, é o Google Shopping que foi criado pelo Craig Nevill-Manning, inicialmente a ferramenta realizava pagamentos para os donos de lojas através do \textit{google adds}, porem em 2012 foi feita uma mudança na ferramenta para que agora as lojas paguem ao google para aparecer no google shopping, agora o google não compara o preço de todas as lojas possíveis somente das que pagam a eles para aparecerem em sua ferramenta.

hoje em dia já surgiram varias ramificações deste produtos, mais a sua essência é a mesma: comparar preços. Existem vários setores diferentes, como por exemplo o site 'Trivago', que faz a comparação de preços de hotes, buscando o valor deste em vários sites de vendas e mostrado a diferença de valor para o usuário poder escolher por onde comprar.